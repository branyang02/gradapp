\documentclass[11pt]{article}
\usepackage[
  a4paper,
  margin=1in,
  headsep=18pt, % separation between header rule and text
]{geometry}
\usepackage{xcolor}
\usepackage{fancyhdr}
\usepackage{microtype}
\usepackage[backend=biber,style=numeric]{biblatex}
\usepackage{setspace}
% \usepackage{mathpazo}
\usepackage{xcolor}
\usepackage{enumitem}
\usepackage{fontawesome}


\definecolor{customred}{HTML}{d14836}


\addbibresource{references.bib}

\singlespacing
% \setstretch{1.5}
\setlength{\parskip}{1em} 

\newcommand{\statement}[1]{\medskip\noindent
  \textcolor{black}{\textbf{#1}}\space
}

\pagestyle{fancy}
\fancyhf{}
\fancyhead[L]{
\textsc{\soptitle}\hfill\footnotesize
\textbf{\yourname}\\
\school\hfill\yourintent}
\fancyfoot[C]{\footnotesize\thepage}

\let\oldcenter\center
\let\oldendcenter\endcenter
\renewenvironment{center}{\setlength\topsep{0pt}\oldcenter}{\oldendcenter}

\newcommand{\wordcount}{
\begin{center}
\rule{0.75\textwidth}{.4pt}\\
{\footnotesize A statement of \numwords \ words}
\end{center}
}

\newif\ifcomments

\newcommand{\comment}[1]{}
\newcommand{\todo}[1]{\ifcomments\textcolor{red}{TODO: #1}\fi}

%%%%%%%%%%%%%%%%%%%%% Edit this section %%%%%%%%%%%%%%%%%%%%

\commentstrue

\newcommand{\soptitle}{Personal Statement}
\newcommand{\yourname}{Brandon Y. Yang}
\newcommand{\yourintent}{\faLink \, \href{https://brandonyifanyang.com/}{brandonyifanyang.com} \quad \faEnvelope \, \href{mailto:branyang@virginia.edu}{branyang@virginia.edu}}

\newcommand{\school}{Cornell University}

%%%%%%%%%%%%%%%%%%%%%%%%%%%%%%%%%%%%%%%%%%%%%%%%%%%%%%%%%%%%

\usepackage[
  breaklinks,
  pdftitle={\yourname - \soptitle},
  pdfauthor={\yourname},
  unicode,
  colorlinks,
]{hyperref}

\begin{document}

I distinctly remember the day my parents dropped me off at Hong Kong International Airport for my first solo trip to the United States. I was 14 years old and about to begin my freshman year at a new boarding school. As I stood in the bustling terminal, a mix of excitement and worry overwhelmed me—not just about navigating the maze of airport terminals, but also about adapting to life in a foreign country without my family. Despite my apprehension, I found my gate, boarded the plane, and landed in Washington, D.C.

However, when I arrived at baggage claim, my suitcase was nowhere to be found. As the conveyor belt emptied, I realized I would need to confront this setback head-on. I mustered my courage and approached the help desk, explaining the situation in my second language. With the airline worker's assistance, I learned that my luggage had been lost and would not arrive for another week. Undeterred, I met the boarding school teacher who had come to pick me up, purchased a few essentials, and embraced the challenge of settling into a completely unfamiliar life.

Living away from home since the age of 14 has shaped me in profound ways. When my mountain bike broke during ninth grade, I turned to YouTube tutorials to figure out how to fix it. When I got sick, I scheduled my own doctor's appointments and followed treatment plans independently. These experiences taught me resourcefulness and self-reliance, forming me into a problem solver who tackles adversity with determination rather than hesitation.

These lessons also helped me find a sense of belonging in high school, despite being thousands of miles from home. Although I arrived knowing no one, I immersed myself in sports teams and clubs, forming lasting friendships. I sought guidance from teachers, building strong connections with mentors who taught me life lessons. As someone who started as an outsider, I deeply value creating communities where everyone feels included and respected, regardless of their background.

This problem-solving mindset also drives my passion for research. While others may find the repetitive nature of experimentation frustrating, I thrive on the challenges it presents. Failed experiments do not discourage me; instead, they fuel my curiosity and drive to dig deeper. The iterative process of testing, troubleshooting, and improving excites me because each step brings me closer to uncovering solutions and gaining new insights. Beyond expanding my knowledge, research has strengthened my analytical thinking and broadened my perspectives.

Graduate school at Cornell is the ideal environment to build on these skills. With the mentorship and resources available, I am eager to grow as a researcher, deepen my expertise, and take on increasingly complex challenges. I look forward to contributing to Cornell's community of inclusion and belonging. My journey so far has taught me the value of perseverance, adaptability, and respect—qualities I will bring to my studies, my field, and the broader academic community.

\end{document}