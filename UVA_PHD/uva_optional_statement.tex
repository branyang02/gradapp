\documentclass[10pt]{article}
\usepackage[
  a4paper,
  margin=1in,
  headsep=18pt, % separation between header rule and text
]{geometry}
\usepackage{xcolor}
\usepackage{fancyhdr}
\usepackage{microtype}
\usepackage[backend=biber,style=numeric]{biblatex}
\usepackage{setspace}
% \usepackage{mathpazo}
\usepackage{xcolor}
\usepackage{enumitem}
\usepackage{fontawesome}


\definecolor{customred}{HTML}{d14836}


\addbibresource{references.bib}

\singlespacing
% \setstretch{1.5}
\setlength{\parskip}{1em} 

\newcommand{\statement}[1]{\medskip\noindent
  \textcolor{black}{\textbf{#1}}\space
}

\pagestyle{fancy}
\fancyhf{}
\fancyhead[L]{
\textsc{\soptitle}\hfill\footnotesize
\textbf{\yourname}\\
\school\hfill\yourintent}
\fancyfoot[C]{\footnotesize\thepage}

\let\oldcenter\center
\let\oldendcenter\endcenter
\renewenvironment{center}{\setlength\topsep{0pt}\oldcenter}{\oldendcenter}

\newcommand{\wordcount}{
\begin{center}
\rule{0.75\textwidth}{.4pt}\\
{\footnotesize A statement of \numwords \ words}
\end{center}
}

\newif\ifcomments

\newcommand{\comment}[1]{}
\newcommand{\todo}[1]{\ifcomments\textcolor{red}{TODO: #1}\fi}

%%%%%%%%%%%%%%%%%%%%% Edit this section %%%%%%%%%%%%%%%%%%%%

\commentstrue

\newcommand{\soptitle}{Optional Statement}
\newcommand{\yourname}{Brandon Y. Yang}
\newcommand{\yourintent}{\faLink \, \href{https://brandonyifanyang.com/}{brandonyifanyang.com} \quad \faEnvelope \, \href{mailto:branyang@virginia.edu}{branyang@virginia.edu}}

\newcommand{\school}{University of Virginia}

%%%%%%%%%%%%%%%%%%%%%%%%%%%%%%%%%%%%%%%%%%%%%%%%%%%%%%%%%%%%

\usepackage[
  breaklinks,
  pdftitle={\yourname - \soptitle},
  pdfauthor={\yourname},
  unicode,
  colorlinks,
]{hyperref}

\begin{document}

\todo{What about your individual background, perspective, or experience will serve as a source of strength for you or those around you at UVA? Feel free to write about any past experience or part of your background that has shaped your perspective and will be a source of strength, including but not limited to those related to your community, upbringing, educational environment, race, gender, or other aspects of your background that are important to you.}

I distinctly remember the day my parents dropped me off at Hong Kong International Airport for my first solo trip to the United States. I was 14 years old and about to begin my freshman year at a new boarding school. As I stood in the bustling terminal, a mix of excitement and worry overwhelmed me—not just about navigating the maze of airport terminals, but also about adapting to life in a foreign country without my family. Despite my apprehension, I found my gate, boarded the plane, and landed in Washington, D.C.

However, when I arrived at baggage claim, my suitcase was nowhere to be found. I waited as the conveyor belt emptied, my belongings nowhere in sight. Summoning my courage, I approached the help desk, explaining the situation in my second language. With the airline worker's assistance, I learned that my luggage had been lost and wouldn't arrive for another week. Undeterred, I met the boarding school teacher who had come to pick me up, purchased a few essentials, and embraced the challenge of settling into a completely unfamiliar life.

Being away from home since the age of 14 has shaped me in profound ways. When my mountain bike broke in ninth grade, I turned to YouTube tutorials to figure out how to fix it. When I got sick, I scheduled my own doctor's appointments and followed treatment plans on my own. These experiences taught me resourcefulness and independence, making me a problem solver who faces adversity with determination rather than hesitation.

This mindset has become central to my passion for research. Where others might find the repetitive nature of experimentation frustrating, I thrive on the challenges it presents. Failed experiments don't discourage me; instead, they fuel my curiosity and drive to dig deeper. The iterative process of testing, troubleshooting, and improving excites me because each step brings me closer to uncovering solutions and gaining new insights.

Beyond expanding my knowledge of my field, research has also equipped me with valuable problem-solving skills. Whether it's resolving unexpected errors or drawing novel connections between ideas, every challenge I tackle broadens my perspective and strengthens my analytical thinking.

Graduate school is the ideal environment for me to continue developing these skills. With the mentorship and resources available, I am eager to grow as a researcher, deepen my expertise, and take on increasingly complex challenges. My journey so far has taught me the value of perseverance and adaptability—qualities I will bring to my studies, my field, and the broader academic community.



\end{document}